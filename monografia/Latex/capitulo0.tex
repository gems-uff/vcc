% Comando para a gera��o da capa
\capa

% Comando para a gera��o da folha de rosto
\folhaderosto

% P�gina do termo de aprova��o
\begin{termo}

% Dentro do ambiente, � necess�rio incluir os membros da banca 
	\banca[Orientador]{Prof. Leonardo Murta, D.Sc.}{UFF}
	\banca[Co-Orientador]{Prof. Alexandre Plastino, D.Sc.}{UFF}
	\banca{Prof. - , M.Sc.}{UFF}
	\banca{Prof. - , D.Sc.}{UFF}	
\end{termo}

% Come�o da p�gina de resumo
\begin{resumo}

VCC
VERTICAL CODE COMPLETION
AQUI ENTRA O RESUMO

% Inclus�o de palavras chave
\palavrasChave{Engenharia de Software, Code Completion, Minera��o de Dados, Minera��o de Padr�es Sequenciais.}

\end{resumo}

% Come�o da p�gina de abstract
\begin{abstract}

VCC
VERTICAL CODE COMPLETION
AQUI ENTRA O ABSTRACT

% Inclus�o de keywords
\keywords{Software Engineering, Code Completion, Data Mining, Sequence Mining.}

\end{abstract}

% P�gina com a lista de acr�nimos
\begin{acronimos}

% Neste ambiente, � poss�vel incluir as siglas e significados
\acronimo{VCC}{\textit{Vertical Code Completion}}
	
\end{acronimos}

% Gera��o do sum�rio
\sumario

% Gera��o da lista de figuras (n�o constar� no sum�rio)
\ProximoForaDoSumario
\listadefiguras

% Gera��o da lista de tabelas (n�o constar� no sum�rio)
\ProximoForaDoSumario
\listadetabelas